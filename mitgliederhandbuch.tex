% RZL-Mitgliederhandbuch
% © 2012 Michael Stapelberg
\documentclass[12pt, DIV16, a4paper]{scrartcl}
\usepackage[utf8]{inputenc}

\usepackage{fancyhdr}
\usepackage{avant}
\usepackage{cmbright} % elegante serifenlose Schrift im Helvetica-Stil

\usepackage{listings}
\usepackage{cmap}
\usepackage{hyphenat}
\usepackage{graphicx}
\usepackage{fixltx2e}
\usepackage{microtype}

%\usepackage[ae]{babel}
%\usepackage[square]{natbib}
\usepackage[pdftex,bookmarks=true,bookmarksnumbered=true,bookmarksopen=true,colorlinks=true,filecolor=black,
                linkcolor=red,urlcolor=blue,plainpages=false,pdfpagelabels,citecolor=black,
                pdftitle={doctitle},pdfauthor={Michael Stapelberg}]{hyperref}
\lstnewenvironment{code}{%
        \lstset{frame=single, language=SQL,%
                showstringspaces=false, basicstyle=\footnotesize\ttfamily}
}{}

\begin{document}
\pagestyle{fancy}
\lhead{RZL}
\chead{Mitgliederhandbuch}
\rhead{2012-03-22}
\newcommand{\np}{\bigskip\noindent}
\setlength{\parindent}{0pt}

\section*{Einleitung}

Dieses Handbuch erklärt dir, wie du dich erfolgreich im RaumZeitLabor
einbringen kannst. Zunächst freuen wir uns, dass du Mitglied im RaumZeitLabor
(geworden) bist!
\np

In einem Hackerspace finden sich viele verschiedene Menschen ein, und das macht
einen großen Teil des Charmes aus. Zum erfolgreichen Zusammenleben sollte man
daher folgenden Grundsatz beachten:

\begin{center}
	Respektiere die anderen Mitglieder, sei freundlich und cool.
\end{center}

Alles Wissenswerte erfährst du in den folgenden Abschnitten.

\section*{Zugang zum RaumZeitLabor}

Als Mitglied erhältst du eine PIN, mit der du die Tür zum RaumZeitLabor
aufschließen kannst. Melde dich dazu in der BenutzerDB an:
\url{https://raumzeitlabor.de/BenutzerDB/register}

Danach kannst Du den Vorstand bitten, dir eine PIN zuzuteilen. Hierzu benötigen
sie deinen Benutzernamen.

\section*{Wöchentliches Treffen}

Die meisten Laboranten triffst du Dienstag abends an. Jede Woche treffen wir
uns zur \textbf{Offenen RaumZeitLaborierung}. Zu dieser Veranstaltung sind
Interessierte ganz besonders eingeladen, da sie immer gut besucht und
interessant ist.
\np

Generell gilt: Sprich uns an, wenn du mit uns reden möchtest, und frag nach,
wenn dich etwas interessiert. Oftmals wirken wir vertieft in Projekte, aber wir
erzählen dir gerne darüber! Manchmal erklären wir zu schnell, oder
unverständlich. Frag bitte nach!

\section*{Mailingliste}

Es gibt eine Mailingliste für das RaumZeitLabor, welche Interessierte abonniert
haben:\\
\url{http://lists.raumzeitlabor.de/mailman/private/raumzeitlabor/}
\np

Auf dieser Mailingliste findest du oftmals Ankündigungen und generelle
Mitteilungen zu allen möglichen Themen. Auch kann es schon einmal hoch her
gehen; sei aber unbesorgt, in der Regel verstehen wir uns alle sehr gut.
\np

Zusätzlich zur ``Haupt''-Mailingliste gibt es auch noch ``info'' sowie
``announce''. Erstere wird als Kontaktadresse auf unseren Seiten verwendet,
damit Anfragen möglichst schnell von den entsprechenden Leuten beantwortet
werden können (möglicherweise also auch von dir!). ``announce'' wird -- wie der
Name bereits sagt -- für öffentliche Ankündigungen verwendet. Hier solltest Du
nur etwas schreiben, wenn Du ein Event von dir ankündigen möchtest. Auch
solltest Du die essentiellen Informationen bereits im Kalender eingetragen
haben.

\section*{Wiki}

In unserem Wiki findet sich Dokumentation zu allem, was sich im RaumZeitLabor
an Equipment befindet (wenn nicht, ergänze es!):
\url{https://raumzeitlabor.de/wiki}

\section*{Internet Relay Chat (IRC)}

Viele Mitglieder finden sich im Internet Relay Chat (IRC) zum lockeren
Quatschen. Verbinde dich mit dem Server \texttt{irc.hackint.net} und leiste uns
im Channel \texttt{\#raumzeitlabor} Gesellschaft. Benutze am besten SSL.
\np

Im IRC kannst du herausfinden, ob das RaumZeitLabor gerade offen ist, indem du
den Befehl \texttt{!raum} schickst. Du kannst die Rundumleuchte im Raum
aktivieren um die Anwesenden auf deine Nachrichten im IRC hinzuweisen, indem du
\texttt{!ping} schickst. Wenn Du hinter \texttt{!ping} noch eine Nachricht
hängt, so wird diese auf dem LED-Display im Raum angezeigt. Anwesende
Laboranten können dir dann aus ein paar vorgegebenen Antwortmöglichkeiten eine
Antwort zukommen lassen.
\np

Zusätzlich gibt es die Möglichkeit herauszufinden, welche Laboranten gerade im
Raum sind. Hierfür gibt es den Befehl \texttt{!weristda}. Der Statusbot
antwortet dir dann mit einer Liste der Mitglieder, deren Geräte in unserem
Netzwerk erreichbar sind. Voraussetzung hierfür ist, dass das jeweilige
Mitglied die MAC-Adressen seiner Geräte in der BenutzerDB eingetragen hat (und
das ist natürlich freiwillig).

\section*{Sourcecode und Projektdateien: Github}

Für Versionskontrolle verwenden wir überwiegend
git\footnote{\url{http://www.git-scm.com/}} und hosten unsere git-repositories
auf \href{http://github.com/}{github} unter der Adresse
\url{http://github.com/raumzeitlabor}. Du kannst ein neues Repository anlegen,
indem du dich an einen der Administratoren auf github wendest, derzeit sind das
u.a.\ sECuRE, Felicitus und Else.

\section*{Vorträge}

Wir laden dich herzlich dazu ein, einen Vortrag zu halten. Ein wichtiges
Konzept im RaumZeitLabor ist Lernen durch Lehren. Ganz egal wozu du etwas
erzählen möchtest, es interessiert mit Sicherheit irgendwen. Auch wenn du
denkst, dass dein Thema doch nicht der Rede wert sei, täuscht das in aller
Regel. Was für dich klar ist, ist vielen neu. Daher: Halte einen Vortrag!
\np

Eine gute Gelegenheit für Vorträge in lockerer Runde ist Dienstags bei der
offenen RaumZeitLaborierung. Wende dich an thinkJD (Jan-Daniel), der das
dienstägliche Programm verwaltet.

\section*{Blog}

Auf \url{http://raumzeitlabor.de/blog} schreiben wir über alle möglichen
Neuigkeiten aus dem RaumZeitLabor. Viele Menschen lesen das Blog und daher wäre
es gut, wenn du neue Beiträge beisteuern könntest -- zum Beispiel, wenn ein
Projekt Fortschritte macht. Melde dich dazu auf \url{http://raumzeitlabor.de/}
an und wende dich an Inte, silsha oder Else um fürs Bloggen freigeschaltet zu
werden.

\end{document}
